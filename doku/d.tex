\documentclass[a4paper, 11pt]{article}
\usepackage[polish]{babel}
\usepackage[MeX]{polski}
\usepackage[utf8]{inputenc}
\usepackage[T1]{fontenc}
%\usepackage{times}
\usepackage{graphicx,wrapfig}
%\usepackage{anysize}
%\usepackage{tikz}
%\usetikzlibrary{calc,through,backgrounds,positioning}
\usepackage{anysize}
\usepackage{float}
%\usepackage{stmaryrd}
%\usepackage{amssymb}
%\usepackage{amsthm}
%\marginsize{3cm}{3cm}{3cm}{3cm}
%\usepackage{amsmath}
%\usepackage{color}
%\usepackage{listings}
%\usepackage{enumerate}

\author{Marcin Jędrzejczyk,Paweł Ogorzały}
\newcommand{\HRule}{\rule{\linewidth}{0.5mm}} % Defines a new command for the horizontal lines, change thickness here
\newtheorem{defi}{Definicja}
\begin{document}
	% \noindent -  w tym akapicie nie bedzie wciecia
	% \ indent - to jest aut., ale powoduje ze jest wciecie
	% \begin{flushleft}, flushright, center - wyrownianie akapitu
	% \textbf{pogrubiany tekst}
	% \textit{kursywa} 
	% 					STRONY 
	%  http://www.codecogs.com/latex/eqneditor.php 
	%  http://www.matematyka.pl/latex.htm
	% 
	\begin{titlepage}
	
	
		
		
		
		\center % Center everything on the page
		
		%----------------------------------------------------------------------------------------
		%	HEADING SECTIONS
		%----------------------------------------------------------------------------------------
		
		\textsc{\LARGE Akademia Górniczo-Hutnicza im. Stanisława Staszica w Krakowie}\\[1.5cm] % Name of your university/college
		\textsc{\Large Studio Projektowe \\ Projekt zaliczeniowy}\\[1.5cm]
		%\textsc{\Large Krak�w}\\[0.5cm] % Major heading such as course name
		%\textsc{\large }\\[0.5cm] % Minor heading such as course title
		
		%----------------------------------------------------------------------------------------
		%	TITLE SECTION
		%----------------------------------------------------------------------------------------
		
		\HRule \\[0.4cm]
		{\fontsize{30}{40}\selectfont Gra mobilna Sudoku na telefony z systemem Android}
		\HRule \\[5.5cm]
		
		%----------------------------------------------------------------------------------------
		%	AUTHOR SECTION
		%----------------------------------------------------------------------------------------
		
		% If you don't want a supervisor, uncomment the two lines below and remove the section above

\begin{minipage}{0.4\textwidth}
\begin{flushleft} \large 
\emph{Autorzy:}\\
Marcin \textsc{Jędrzejczyk}\\ 
Paweł \textsc{Ogorzały}
\end{flushleft}
\end{minipage}
~
\begin{minipage}{0.4\textwidth}
\begin{flushright} \large
\emph{Opiekun:}\\
 Dr inż. Maciej \textsc{Szymkat}  % Supervisor's Name
\end{flushright}
\end{minipage} \\[5cm]

		
		%----------------------------------------------------------------------------------------
		%	DATE SECTION
		%----------------------------------------------------------------------------------------
		
		{\large \today}\\[3cm] % Date, change the \today to a set date if you want to be precise
		
		%----------------------------------------------------------------------------------------
		%	LOGO SECTION
		%----------------------------------------------------------------------------------------
		
		%\includegraphics{Logo}\\[1cm] % Include a department/university logo - this will require the graphicx package
		
		%----------------------------------------------------------------------------------------
		
		\vfill % Fill the rest of the page with whitespace
		
	\end{titlepage}
	
	\newpage
	
	\tableofcontents
	\newpage
	
	\listoffigures
	\newpage
	
	
	
	
	\section{Sudoku}
	\subsection{Co to Sudoku}
W Sudoku gra się na planszy o wymiarach 9x9 podzielonej na mniejsze \"obszary\" o wymiarach 3x3.  Na początku gry niektóre z pól planszy Sudoku są już wypełnione liczbami.  Celem gry jest uzupełnienie pozostałych pól planszy cyframi od 1 do 9 (po jednej cyfrze w każdym polu) przy zachowaniu następujących reguł:
\begin{itemize}
\item Każda cyfra może się pojawić tylko raz w każdym wierszu,
\item Każda cyfra może się pojawić tylko raz w każdej kolumnie,
\item Każda cyfra może się pojawić tylko raz w każdym obszarze.
\end{itemize}

	
	% latex(szybka kompilacja),bibtex,kompilacja,kompilacja
%\bibliographystyle{plain}% plain/abbrv/alpha
%\bibliography{bibliografia}%plik .bib
\end{document}


